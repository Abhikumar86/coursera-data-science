\documentclass[]{article}
\usepackage{lmodern}
\usepackage{amssymb,amsmath}
\usepackage{ifxetex,ifluatex}
\usepackage{fixltx2e} % provides \textsubscript
\ifnum 0\ifxetex 1\fi\ifluatex 1\fi=0 % if pdftex
  \usepackage[T1]{fontenc}
  \usepackage[utf8]{inputenc}
\else % if luatex or xelatex
  \ifxetex
    \usepackage{mathspec}
  \else
    \usepackage{fontspec}
  \fi
  \defaultfontfeatures{Ligatures=TeX,Scale=MatchLowercase}
\fi
% use upquote if available, for straight quotes in verbatim environments
\IfFileExists{upquote.sty}{\usepackage{upquote}}{}
% use microtype if available
\IfFileExists{microtype.sty}{%
\usepackage[]{microtype}
\UseMicrotypeSet[protrusion]{basicmath} % disable protrusion for tt fonts
}{}
\PassOptionsToPackage{hyphens}{url} % url is loaded by hyperref
\usepackage[unicode=true]{hyperref}
\hypersetup{
            pdftitle={Identifying the storm event type that causes highest human health and economic damage in US},
            pdfauthor={Abhishek Kumar},
            pdfborder={0 0 0},
            breaklinks=true}
\urlstyle{same}  % don't use monospace font for urls
\usepackage[margin=1in]{geometry}
\usepackage{color}
\usepackage{fancyvrb}
\newcommand{\VerbBar}{|}
\newcommand{\VERB}{\Verb[commandchars=\\\{\}]}
\DefineVerbatimEnvironment{Highlighting}{Verbatim}{commandchars=\\\{\}}
% Add ',fontsize=\small' for more characters per line
\usepackage{framed}
\definecolor{shadecolor}{RGB}{248,248,248}
\newenvironment{Shaded}{\begin{snugshade}}{\end{snugshade}}
\newcommand{\KeywordTok}[1]{\textcolor[rgb]{0.13,0.29,0.53}{\textbf{#1}}}
\newcommand{\DataTypeTok}[1]{\textcolor[rgb]{0.13,0.29,0.53}{#1}}
\newcommand{\DecValTok}[1]{\textcolor[rgb]{0.00,0.00,0.81}{#1}}
\newcommand{\BaseNTok}[1]{\textcolor[rgb]{0.00,0.00,0.81}{#1}}
\newcommand{\FloatTok}[1]{\textcolor[rgb]{0.00,0.00,0.81}{#1}}
\newcommand{\ConstantTok}[1]{\textcolor[rgb]{0.00,0.00,0.00}{#1}}
\newcommand{\CharTok}[1]{\textcolor[rgb]{0.31,0.60,0.02}{#1}}
\newcommand{\SpecialCharTok}[1]{\textcolor[rgb]{0.00,0.00,0.00}{#1}}
\newcommand{\StringTok}[1]{\textcolor[rgb]{0.31,0.60,0.02}{#1}}
\newcommand{\VerbatimStringTok}[1]{\textcolor[rgb]{0.31,0.60,0.02}{#1}}
\newcommand{\SpecialStringTok}[1]{\textcolor[rgb]{0.31,0.60,0.02}{#1}}
\newcommand{\ImportTok}[1]{#1}
\newcommand{\CommentTok}[1]{\textcolor[rgb]{0.56,0.35,0.01}{\textit{#1}}}
\newcommand{\DocumentationTok}[1]{\textcolor[rgb]{0.56,0.35,0.01}{\textbf{\textit{#1}}}}
\newcommand{\AnnotationTok}[1]{\textcolor[rgb]{0.56,0.35,0.01}{\textbf{\textit{#1}}}}
\newcommand{\CommentVarTok}[1]{\textcolor[rgb]{0.56,0.35,0.01}{\textbf{\textit{#1}}}}
\newcommand{\OtherTok}[1]{\textcolor[rgb]{0.56,0.35,0.01}{#1}}
\newcommand{\FunctionTok}[1]{\textcolor[rgb]{0.00,0.00,0.00}{#1}}
\newcommand{\VariableTok}[1]{\textcolor[rgb]{0.00,0.00,0.00}{#1}}
\newcommand{\ControlFlowTok}[1]{\textcolor[rgb]{0.13,0.29,0.53}{\textbf{#1}}}
\newcommand{\OperatorTok}[1]{\textcolor[rgb]{0.81,0.36,0.00}{\textbf{#1}}}
\newcommand{\BuiltInTok}[1]{#1}
\newcommand{\ExtensionTok}[1]{#1}
\newcommand{\PreprocessorTok}[1]{\textcolor[rgb]{0.56,0.35,0.01}{\textit{#1}}}
\newcommand{\AttributeTok}[1]{\textcolor[rgb]{0.77,0.63,0.00}{#1}}
\newcommand{\RegionMarkerTok}[1]{#1}
\newcommand{\InformationTok}[1]{\textcolor[rgb]{0.56,0.35,0.01}{\textbf{\textit{#1}}}}
\newcommand{\WarningTok}[1]{\textcolor[rgb]{0.56,0.35,0.01}{\textbf{\textit{#1}}}}
\newcommand{\AlertTok}[1]{\textcolor[rgb]{0.94,0.16,0.16}{#1}}
\newcommand{\ErrorTok}[1]{\textcolor[rgb]{0.64,0.00,0.00}{\textbf{#1}}}
\newcommand{\NormalTok}[1]{#1}
\usepackage{graphicx,grffile}
\makeatletter
\def\maxwidth{\ifdim\Gin@nat@width>\linewidth\linewidth\else\Gin@nat@width\fi}
\def\maxheight{\ifdim\Gin@nat@height>\textheight\textheight\else\Gin@nat@height\fi}
\makeatother
% Scale images if necessary, so that they will not overflow the page
% margins by default, and it is still possible to overwrite the defaults
% using explicit options in \includegraphics[width, height, ...]{}
\setkeys{Gin}{width=\maxwidth,height=\maxheight,keepaspectratio}
\IfFileExists{parskip.sty}{%
\usepackage{parskip}
}{% else
\setlength{\parindent}{0pt}
\setlength{\parskip}{6pt plus 2pt minus 1pt}
}
\setlength{\emergencystretch}{3em}  % prevent overfull lines
\providecommand{\tightlist}{%
  \setlength{\itemsep}{0pt}\setlength{\parskip}{0pt}}
\setcounter{secnumdepth}{0}
% Redefines (sub)paragraphs to behave more like sections
\ifx\paragraph\undefined\else
\let\oldparagraph\paragraph
\renewcommand{\paragraph}[1]{\oldparagraph{#1}\mbox{}}
\fi
\ifx\subparagraph\undefined\else
\let\oldsubparagraph\subparagraph
\renewcommand{\subparagraph}[1]{\oldsubparagraph{#1}\mbox{}}
\fi

% set default figure placement to htbp
\makeatletter
\def\fps@figure{htbp}
\makeatother


\title{Identifying the storm event type that causes highest human health and
economic damage in US}
\author{Abhishek Kumar}
\date{16 July 2020}

\begin{document}
\maketitle

\section{Exploratory Data Analysis with NOAA storm
data}\label{exploratory-data-analysis-with-noaa-storm-data}

\section{1. Synopsis}\label{synopsis}

Storms and other severe weather events can cause both public health and
economic problems for communities and municipalities. Many severe events
can result in fatalities, injuries, and property damage, and preventing
such outcomes to the extent possible is a key concern.

\section{2. Data}\label{data}

This analysis involves an exploration of the U.S. National Oceanic and
Atmospheric Administration's (NOAA) storm database. This data is
available in the form of a comma-separated-value file compressed via the
bzip2 algorithm to reduce its size. This data can be downloaded from the
following link:

\begin{itemize}
\tightlist
\item
  \href{https://d396qusza40orc.cloudfront.net/repdata\%2Fdata\%2FStormData.csv.bz2}{Storm
  Data} {[}47Mb{]}
\end{itemize}

The documentation on how some of the variables are constructed/defined
for the database is available from below links:

\begin{itemize}
\item
  National Weather Service
  \href{https://d396qusza40orc.cloudfront.net/repdata\%2Fpeer2_doc\%2Fpd01016005curr.pdf}{Storm
  Data Documentation}
\item
  National Climatic Data Center Storm Events
  \href{https://d396qusza40orc.cloudfront.net/repdata\%2Fpeer2_doc\%2FNCDC\%20Storm\%20Events-FAQ\%20Page.pdf}{FAQ}
\end{itemize}

This database tracks characteristics of major storms and weather events
in the United States, including when and where they occur, as well as
estimates of any fatalities, injuries, and property damage. The events
in the database start in the year 1950 and end in November 2011. In the
earlier years of the database there are generally fewer events recorded,
most likely due to a lack of good records. More recent years should be
considered more complete.

\section{3. Data Processing}\label{data-processing}

Firstly, load sone of the required packages using the \texttt{library()}
function

\begin{Shaded}
\begin{Highlighting}[]
\KeywordTok{library}\NormalTok{(dplyr)}
\end{Highlighting}
\end{Shaded}

\begin{verbatim}
## 
## Attaching package: 'dplyr'
\end{verbatim}

\begin{verbatim}
## The following objects are masked from 'package:stats':
## 
##     filter, lag
\end{verbatim}

\begin{verbatim}
## The following objects are masked from 'package:base':
## 
##     intersect, setdiff, setequal, union
\end{verbatim}

\begin{Shaded}
\begin{Highlighting}[]
\KeywordTok{library}\NormalTok{(ggplot2)}
\end{Highlighting}
\end{Shaded}

\begin{verbatim}
## Warning: package 'ggplot2' was built under R version 4.0.2
\end{verbatim}

\begin{Shaded}
\begin{Highlighting}[]
\KeywordTok{library}\NormalTok{(tidyr)}
\end{Highlighting}
\end{Shaded}

\subsection{3.1 Download the data}\label{download-the-data}

Firstly, the data was downloaded using the following commands:

\begin{Shaded}
\begin{Highlighting}[]
\ControlFlowTok{if}\NormalTok{(}\OperatorTok{!}\KeywordTok{file.exists}\NormalTok{(}\StringTok{"StormData.csv.bz2"}\NormalTok{)) \{}
\NormalTok{      fileUrl <-}\StringTok{ "https://d396qusza40orc.cloudfront.net/repdata%2Fdata%2FStormData.csv.bz2"}
      \KeywordTok{download.file}\NormalTok{(fileUrl, }\DataTypeTok{destfile =} \StringTok{"./StormData.csv.bz2"}\NormalTok{)}
\NormalTok{      \}}
\end{Highlighting}
\end{Shaded}

\subsection{3.2 Loading the data}\label{loading-the-data}

Since, this data is available in the form of a comma-separated-value
file compressed via the bzip2 algorithm to reduce its size. The function
\texttt{read.csv()} can read the \texttt{.csv.bz2} file

\begin{Shaded}
\begin{Highlighting}[]
\NormalTok{storm.data <-}\StringTok{ }\KeywordTok{read.csv}\NormalTok{(}\DataTypeTok{file =} \StringTok{"StormData.csv.bz2"}\NormalTok{, }\DataTypeTok{header =} \OtherTok{TRUE}\NormalTok{, }\DataTypeTok{sep =} \StringTok{","}\NormalTok{)}
\KeywordTok{head}\NormalTok{(storm.data)}
\end{Highlighting}
\end{Shaded}

\begin{verbatim}
##   STATE__           BGN_DATE BGN_TIME TIME_ZONE COUNTY COUNTYNAME STATE  EVTYPE
## 1       1  4/18/1950 0:00:00     0130       CST     97     MOBILE    AL TORNADO
## 2       1  4/18/1950 0:00:00     0145       CST      3    BALDWIN    AL TORNADO
## 3       1  2/20/1951 0:00:00     1600       CST     57    FAYETTE    AL TORNADO
## 4       1   6/8/1951 0:00:00     0900       CST     89    MADISON    AL TORNADO
## 5       1 11/15/1951 0:00:00     1500       CST     43    CULLMAN    AL TORNADO
## 6       1 11/15/1951 0:00:00     2000       CST     77 LAUDERDALE    AL TORNADO
##   BGN_RANGE BGN_AZI BGN_LOCATI END_DATE END_TIME COUNTY_END COUNTYENDN
## 1         0                                               0         NA
## 2         0                                               0         NA
## 3         0                                               0         NA
## 4         0                                               0         NA
## 5         0                                               0         NA
## 6         0                                               0         NA
##   END_RANGE END_AZI END_LOCATI LENGTH WIDTH F MAG FATALITIES INJURIES PROPDMG
## 1         0                      14.0   100 3   0          0       15    25.0
## 2         0                       2.0   150 2   0          0        0     2.5
## 3         0                       0.1   123 2   0          0        2    25.0
## 4         0                       0.0   100 2   0          0        2     2.5
## 5         0                       0.0   150 2   0          0        2     2.5
## 6         0                       1.5   177 2   0          0        6     2.5
##   PROPDMGEXP CROPDMG CROPDMGEXP WFO STATEOFFIC ZONENAMES LATITUDE LONGITUDE
## 1          K       0                                         3040      8812
## 2          K       0                                         3042      8755
## 3          K       0                                         3340      8742
## 4          K       0                                         3458      8626
## 5          K       0                                         3412      8642
## 6          K       0                                         3450      8748
##   LATITUDE_E LONGITUDE_ REMARKS REFNUM
## 1       3051       8806              1
## 2          0          0              2
## 3          0          0              3
## 4          0          0              4
## 5          0          0              5
## 6          0          0              6
\end{verbatim}

After loading the data, the structure of data can be viewed using the
\texttt{str()} function

\begin{Shaded}
\begin{Highlighting}[]
\KeywordTok{str}\NormalTok{(storm.data)}
\end{Highlighting}
\end{Shaded}

\begin{verbatim}
## 'data.frame':    902297 obs. of  37 variables:
##  $ STATE__   : num  1 1 1 1 1 1 1 1 1 1 ...
##  $ BGN_DATE  : chr  "4/18/1950 0:00:00" "4/18/1950 0:00:00" "2/20/1951 0:00:00" "6/8/1951 0:00:00" ...
##  $ BGN_TIME  : chr  "0130" "0145" "1600" "0900" ...
##  $ TIME_ZONE : chr  "CST" "CST" "CST" "CST" ...
##  $ COUNTY    : num  97 3 57 89 43 77 9 123 125 57 ...
##  $ COUNTYNAME: chr  "MOBILE" "BALDWIN" "FAYETTE" "MADISON" ...
##  $ STATE     : chr  "AL" "AL" "AL" "AL" ...
##  $ EVTYPE    : chr  "TORNADO" "TORNADO" "TORNADO" "TORNADO" ...
##  $ BGN_RANGE : num  0 0 0 0 0 0 0 0 0 0 ...
##  $ BGN_AZI   : chr  "" "" "" "" ...
##  $ BGN_LOCATI: chr  "" "" "" "" ...
##  $ END_DATE  : chr  "" "" "" "" ...
##  $ END_TIME  : chr  "" "" "" "" ...
##  $ COUNTY_END: num  0 0 0 0 0 0 0 0 0 0 ...
##  $ COUNTYENDN: logi  NA NA NA NA NA NA ...
##  $ END_RANGE : num  0 0 0 0 0 0 0 0 0 0 ...
##  $ END_AZI   : chr  "" "" "" "" ...
##  $ END_LOCATI: chr  "" "" "" "" ...
##  $ LENGTH    : num  14 2 0.1 0 0 1.5 1.5 0 3.3 2.3 ...
##  $ WIDTH     : num  100 150 123 100 150 177 33 33 100 100 ...
##  $ F         : int  3 2 2 2 2 2 2 1 3 3 ...
##  $ MAG       : num  0 0 0 0 0 0 0 0 0 0 ...
##  $ FATALITIES: num  0 0 0 0 0 0 0 0 1 0 ...
##  $ INJURIES  : num  15 0 2 2 2 6 1 0 14 0 ...
##  $ PROPDMG   : num  25 2.5 25 2.5 2.5 2.5 2.5 2.5 25 25 ...
##  $ PROPDMGEXP: chr  "K" "K" "K" "K" ...
##  $ CROPDMG   : num  0 0 0 0 0 0 0 0 0 0 ...
##  $ CROPDMGEXP: chr  "" "" "" "" ...
##  $ WFO       : chr  "" "" "" "" ...
##  $ STATEOFFIC: chr  "" "" "" "" ...
##  $ ZONENAMES : chr  "" "" "" "" ...
##  $ LATITUDE  : num  3040 3042 3340 3458 3412 ...
##  $ LONGITUDE : num  8812 8755 8742 8626 8642 ...
##  $ LATITUDE_E: num  3051 0 0 0 0 ...
##  $ LONGITUDE_: num  8806 0 0 0 0 ...
##  $ REMARKS   : chr  "" "" "" "" ...
##  $ REFNUM    : num  1 2 3 4 5 6 7 8 9 10 ...
\end{verbatim}

So, this data is a dataframe with 902297 observations (rows) and 37
variables (columns). Also, the date and time are represented as
characters, which need to be converted into Date and Time format

\subsection{3.3 Preprocessiog}\label{preprocessiog}

Lets convert the date and time

\begin{Shaded}
\begin{Highlighting}[]
\NormalTok{storm.data}\OperatorTok{$}\NormalTok{BGN_DATE <-}\StringTok{ }\KeywordTok{as.Date}\NormalTok{(storm.data}\OperatorTok{$}\NormalTok{BGN_DATE, }\StringTok{"%m/%d/%Y"}\NormalTok{)}
\KeywordTok{str}\NormalTok{(storm.data)}
\end{Highlighting}
\end{Shaded}

\begin{verbatim}
## 'data.frame':    902297 obs. of  37 variables:
##  $ STATE__   : num  1 1 1 1 1 1 1 1 1 1 ...
##  $ BGN_DATE  : Date, format: "1950-04-18" "1950-04-18" ...
##  $ BGN_TIME  : chr  "0130" "0145" "1600" "0900" ...
##  $ TIME_ZONE : chr  "CST" "CST" "CST" "CST" ...
##  $ COUNTY    : num  97 3 57 89 43 77 9 123 125 57 ...
##  $ COUNTYNAME: chr  "MOBILE" "BALDWIN" "FAYETTE" "MADISON" ...
##  $ STATE     : chr  "AL" "AL" "AL" "AL" ...
##  $ EVTYPE    : chr  "TORNADO" "TORNADO" "TORNADO" "TORNADO" ...
##  $ BGN_RANGE : num  0 0 0 0 0 0 0 0 0 0 ...
##  $ BGN_AZI   : chr  "" "" "" "" ...
##  $ BGN_LOCATI: chr  "" "" "" "" ...
##  $ END_DATE  : chr  "" "" "" "" ...
##  $ END_TIME  : chr  "" "" "" "" ...
##  $ COUNTY_END: num  0 0 0 0 0 0 0 0 0 0 ...
##  $ COUNTYENDN: logi  NA NA NA NA NA NA ...
##  $ END_RANGE : num  0 0 0 0 0 0 0 0 0 0 ...
##  $ END_AZI   : chr  "" "" "" "" ...
##  $ END_LOCATI: chr  "" "" "" "" ...
##  $ LENGTH    : num  14 2 0.1 0 0 1.5 1.5 0 3.3 2.3 ...
##  $ WIDTH     : num  100 150 123 100 150 177 33 33 100 100 ...
##  $ F         : int  3 2 2 2 2 2 2 1 3 3 ...
##  $ MAG       : num  0 0 0 0 0 0 0 0 0 0 ...
##  $ FATALITIES: num  0 0 0 0 0 0 0 0 1 0 ...
##  $ INJURIES  : num  15 0 2 2 2 6 1 0 14 0 ...
##  $ PROPDMG   : num  25 2.5 25 2.5 2.5 2.5 2.5 2.5 25 25 ...
##  $ PROPDMGEXP: chr  "K" "K" "K" "K" ...
##  $ CROPDMG   : num  0 0 0 0 0 0 0 0 0 0 ...
##  $ CROPDMGEXP: chr  "" "" "" "" ...
##  $ WFO       : chr  "" "" "" "" ...
##  $ STATEOFFIC: chr  "" "" "" "" ...
##  $ ZONENAMES : chr  "" "" "" "" ...
##  $ LATITUDE  : num  3040 3042 3340 3458 3412 ...
##  $ LONGITUDE : num  8812 8755 8742 8626 8642 ...
##  $ LATITUDE_E: num  3051 0 0 0 0 ...
##  $ LONGITUDE_: num  8806 0 0 0 0 ...
##  $ REMARKS   : chr  "" "" "" "" ...
##  $ REFNUM    : num  1 2 3 4 5 6 7 8 9 10 ...
\end{verbatim}

Now, date has been converted. Let's look at our questions

\begin{itemize}
\item
  \emph{Across the United States, which types of events (as indicated in
  the \texttt{EVTYPE} variable) are most harmful with respect to
  population health?}
\item
  \emph{Across the United States, which types of events have the
  greatest economic consequences?}
\end{itemize}

So we need to focus on health and economic damage, therefore we do not
need to process all the data, we can subset the data by selecting
desirable columns.

\begin{itemize}
\tightlist
\item
  Health: FATALITIES and INJURIES
\item
  Economic: PROPDMG, PROPDMGEXP, CROPDMG, CROPDMGEXP
\end{itemize}

After selecting the desirable variables, we may filter the observations,
which have some impacts i.e.~values greater than zero. This will results
in reduced data for analysis.

\begin{Shaded}
\begin{Highlighting}[]
\NormalTok{storm.damage <-}\StringTok{ }\NormalTok{storm.data }\OperatorTok\StringTok{ }
\StringTok{      }\KeywordTok{select}\NormalTok{(EVTYPE, FATALITIES, INJURIES, PROPDMG, PROPDMGEXP, CROPDMG, CROPDMGEXP)}
\end{Highlighting}
\end{Shaded}

Considering the given data, human health can be assessed by two
variables i.e. FATALITIES and INJURIES. So for achieving this, * we need
to first filter the data with storm events which had some impact on
human health * Then, the data is grouped by EVent types * It is followed
by summarising data to sum the FATALITIES and INJURIES with EVTYPE. *
The data is then arranged in decreasing values of Fatalities and
Injuries. * This can be achieved by using either the
\texttt{aggregate()} function or using the package \texttt{dplyr}. Here,
I am using the the package \texttt{dplyr} * After this, the data was
tranformed into a longer format to plot with different panels

\begin{Shaded}
\begin{Highlighting}[]
\NormalTok{fat <-}\StringTok{ }\NormalTok{storm.damage }\OperatorTok\StringTok{ }
\StringTok{      }\KeywordTok{filter}\NormalTok{(FATALITIES }\OperatorTok{>}\StringTok{ }\DecValTok{0} \OperatorTok{|}\StringTok{ }\NormalTok{INJURIES }\OperatorTok{>}\StringTok{ }\DecValTok{0}\NormalTok{) }\OperatorTok
\StringTok{      }\KeywordTok{group_by}\NormalTok{(EVTYPE) }\OperatorTok\StringTok{ }
\StringTok{      }\KeywordTok{summarise}\NormalTok{(}\DataTypeTok{Fatalities =} \KeywordTok{sum}\NormalTok{(FATALITIES)) }\OperatorTok
\StringTok{      }\KeywordTok{arrange}\NormalTok{(}\KeywordTok{desc}\NormalTok{(Fatalities))}
\end{Highlighting}
\end{Shaded}

\begin{verbatim}
## `summarise()` ungrouping output (override with `.groups` argument)
\end{verbatim}

\begin{Shaded}
\begin{Highlighting}[]
\NormalTok{inj <-}\StringTok{ }\NormalTok{storm.damage }\OperatorTok\StringTok{ }
\StringTok{      }\KeywordTok{filter}\NormalTok{(FATALITIES }\OperatorTok{>}\StringTok{ }\DecValTok{0} \OperatorTok{|}\StringTok{ }\NormalTok{INJURIES }\OperatorTok{>}\StringTok{ }\DecValTok{0}\NormalTok{) }\OperatorTok
\StringTok{      }\KeywordTok{group_by}\NormalTok{(EVTYPE) }\OperatorTok\StringTok{ }
\StringTok{      }\KeywordTok{summarise}\NormalTok{(}\DataTypeTok{Injuries =} \KeywordTok{sum}\NormalTok{(INJURIES)) }\OperatorTok
\StringTok{      }\KeywordTok{arrange}\NormalTok{(}\KeywordTok{desc}\NormalTok{(Injuries))}
\end{Highlighting}
\end{Shaded}

\begin{verbatim}
## `summarise()` ungrouping output (override with `.groups` argument)
\end{verbatim}

\begin{Shaded}
\begin{Highlighting}[]
\NormalTok{health.Dam <-}\StringTok{ }\KeywordTok{merge}\NormalTok{(fat[}\DecValTok{1}\OperatorTok{:}\DecValTok{10}\NormalTok{,], inj[}\DecValTok{1}\OperatorTok{:}\DecValTok{10}\NormalTok{,]) }\OperatorTok
\StringTok{      }\KeywordTok{pivot_longer}\NormalTok{(}\OperatorTok{-}\NormalTok{EVTYPE, }\DataTypeTok{names_to =} \StringTok{"variables"}\NormalTok{, }\DataTypeTok{values_to =} \StringTok{"values"}\NormalTok{)}
\end{Highlighting}
\end{Shaded}

Our Second questions was:

\begin{itemize}
\tightlist
\item
  \emph{Across the United States, which types of events have the
  greatest economic consequences?}
\end{itemize}

So, for economid damage, we have already selected variables and stored
in storm.damage dataframe.There are two types of variables that impact
the economic damage i.e.~Crop damage and property damage. However, this
data has some exponents which are defined in CROPDAMEXP and PROPDAMEXP,
respectively. So first we have decode these exponents. Lets look at
unique varibales in EXPs

\begin{Shaded}
\begin{Highlighting}[]
\KeywordTok{unique}\NormalTok{(storm.damage}\OperatorTok{$}\NormalTok{CROPDMGEXP)}
\end{Highlighting}
\end{Shaded}

\begin{verbatim}
## [1] ""  "M" "K" "m" "B" "?" "0" "k" "2"
\end{verbatim}

\begin{Shaded}
\begin{Highlighting}[]
\KeywordTok{unique}\NormalTok{(storm.damage}\OperatorTok{$}\NormalTok{PROPDMGEXP)}
\end{Highlighting}
\end{Shaded}

\begin{verbatim}
##  [1] "K" "M" ""  "B" "m" "+" "0" "5" "6" "?" "4" "2" "3" "h" "7" "H" "-" "1" "8"
\end{verbatim}

Here, these symbols represent different exponents which need to be
replaced in the data. * 0 = 1 * 1 = 10 * 2 = 100 * 3 = 1,000 * 4 =
10,000 * 5 = 100,000 * 6 = 1,000,000 * 7 = 10,000,000 * 8 = 100,000,000
* 9 = 1,000,000,000 * H = 100 * h = 100 * K = 1,000 * k = 1,000 * M =
1,000,000 * m = 1,000,000 * B = 1,000,000,000 * + = 1 * ``'' = 1 * - = 1

\begin{Shaded}
\begin{Highlighting}[]
\NormalTok{storm.damage}\OperatorTok{$}\NormalTok{CropFactor[(storm.damage}\OperatorTok{$}\NormalTok{CROPDMGEXP }\OperatorTok{==}\StringTok{ ""}\NormalTok{)] <-}\StringTok{ }\DecValTok{10}\OperatorTok{^}\DecValTok{0}
\NormalTok{storm.damage}\OperatorTok{$}\NormalTok{CropFactor[(storm.damage}\OperatorTok{$}\NormalTok{CROPDMGEXP }\OperatorTok{==}\StringTok{ "M"}\NormalTok{)] <-}\StringTok{ }\DecValTok{10}\OperatorTok{^}\DecValTok{6}
\NormalTok{storm.damage}\OperatorTok{$}\NormalTok{CropFactor[(storm.damage}\OperatorTok{$}\NormalTok{CROPDMGEXP }\OperatorTok{==}\StringTok{ "K"}\NormalTok{)] <-}\StringTok{ }\DecValTok{10}\OperatorTok{^}\DecValTok{3}
\NormalTok{storm.damage}\OperatorTok{$}\NormalTok{CropFactor[(storm.damage}\OperatorTok{$}\NormalTok{CROPDMGEXP }\OperatorTok{==}\StringTok{ "m"}\NormalTok{)] <-}\StringTok{ }\DecValTok{10}\OperatorTok{^}\DecValTok{6}
\NormalTok{storm.damage}\OperatorTok{$}\NormalTok{CropFactor[(storm.damage}\OperatorTok{$}\NormalTok{CROPDMGEXP }\OperatorTok{==}\StringTok{ "B"}\NormalTok{)] <-}\StringTok{ }\DecValTok{10}\OperatorTok{^}\DecValTok{9}
\NormalTok{storm.damage}\OperatorTok{$}\NormalTok{CropFactor[(storm.damage}\OperatorTok{$}\NormalTok{CROPDMGEXP }\OperatorTok{==}\StringTok{ "?"}\NormalTok{)] <-}\StringTok{ }\DecValTok{10}\OperatorTok{^}\DecValTok{0}
\NormalTok{storm.damage}\OperatorTok{$}\NormalTok{CropFactor[(storm.damage}\OperatorTok{$}\NormalTok{CROPDMGEXP }\OperatorTok{==}\StringTok{ "0"}\NormalTok{)] <-}\StringTok{ }\DecValTok{10}\OperatorTok{^}\DecValTok{0}
\NormalTok{storm.damage}\OperatorTok{$}\NormalTok{CropFactor[(storm.damage}\OperatorTok{$}\NormalTok{CROPDMGEXP }\OperatorTok{==}\StringTok{ "k"}\NormalTok{)] <-}\StringTok{ }\DecValTok{10}\OperatorTok{^}\DecValTok{3}
\NormalTok{storm.damage}\OperatorTok{$}\NormalTok{CropFactor[(storm.damage}\OperatorTok{$}\NormalTok{CROPDMGEXP }\OperatorTok{==}\StringTok{ "2"}\NormalTok{)] <-}\StringTok{ }\DecValTok{10}\OperatorTok{^}\DecValTok{2}

\NormalTok{storm.damage}\OperatorTok{$}\NormalTok{PropFactor[(storm.damage}\OperatorTok{$}\NormalTok{PROPDMGEXP }\OperatorTok{==}\StringTok{ "K"}\NormalTok{)] <-}\StringTok{ }\DecValTok{10}\OperatorTok{^}\DecValTok{3}
\NormalTok{storm.damage}\OperatorTok{$}\NormalTok{PropFactor[(storm.damage}\OperatorTok{$}\NormalTok{PROPDMGEXP }\OperatorTok{==}\StringTok{ "M"}\NormalTok{)] <-}\StringTok{ }\DecValTok{10}\OperatorTok{^}\DecValTok{6}
\NormalTok{storm.damage}\OperatorTok{$}\NormalTok{PropFactor[(storm.damage}\OperatorTok{$}\NormalTok{PROPDMGEXP }\OperatorTok{==}\StringTok{ ""}\NormalTok{)] <-}\StringTok{ }\DecValTok{10}\OperatorTok{^}\DecValTok{0}
\NormalTok{storm.damage}\OperatorTok{$}\NormalTok{PropFactor[(storm.damage}\OperatorTok{$}\NormalTok{PROPDMGEXP }\OperatorTok{==}\StringTok{ "B"}\NormalTok{)] <-}\StringTok{ }\DecValTok{10}\OperatorTok{^}\DecValTok{9}
\NormalTok{storm.damage}\OperatorTok{$}\NormalTok{PropFactor[(storm.damage}\OperatorTok{$}\NormalTok{PROPDMGEXP }\OperatorTok{==}\StringTok{ "m"}\NormalTok{)] <-}\StringTok{ }\DecValTok{10}\OperatorTok{^}\DecValTok{6}
\NormalTok{storm.damage}\OperatorTok{$}\NormalTok{PropFactor[(storm.damage}\OperatorTok{$}\NormalTok{PROPDMGEXP }\OperatorTok{==}\StringTok{ "+"}\NormalTok{)] <-}\StringTok{ }\DecValTok{10}\OperatorTok{^}\DecValTok{0}
\NormalTok{storm.damage}\OperatorTok{$}\NormalTok{PropFactor[(storm.damage}\OperatorTok{$}\NormalTok{PROPDMGEXP }\OperatorTok{==}\StringTok{ "0"}\NormalTok{)] <-}\StringTok{ }\DecValTok{10}\OperatorTok{^}\DecValTok{0}
\NormalTok{storm.damage}\OperatorTok{$}\NormalTok{PropFactor[(storm.damage}\OperatorTok{$}\NormalTok{PROPDMGEXP }\OperatorTok{==}\StringTok{ "5"}\NormalTok{)] <-}\StringTok{ }\DecValTok{10}\OperatorTok{^}\DecValTok{5}
\NormalTok{storm.damage}\OperatorTok{$}\NormalTok{PropFactor[(storm.damage}\OperatorTok{$}\NormalTok{PROPDMGEXP }\OperatorTok{==}\StringTok{ "6"}\NormalTok{)] <-}\StringTok{ }\DecValTok{10}\OperatorTok{^}\DecValTok{6}
\NormalTok{storm.damage}\OperatorTok{$}\NormalTok{PropFactor[(storm.damage}\OperatorTok{$}\NormalTok{PROPDMGEXP }\OperatorTok{==}\StringTok{ "?"}\NormalTok{)] <-}\StringTok{ }\DecValTok{10}\OperatorTok{^}\DecValTok{0}
\NormalTok{storm.damage}\OperatorTok{$}\NormalTok{PropFactor[(storm.damage}\OperatorTok{$}\NormalTok{PROPDMGEXP }\OperatorTok{==}\StringTok{ "4"}\NormalTok{)] <-}\StringTok{ }\DecValTok{10}\OperatorTok{^}\DecValTok{4}
\NormalTok{storm.damage}\OperatorTok{$}\NormalTok{PropFactor[(storm.damage}\OperatorTok{$}\NormalTok{PROPDMGEXP }\OperatorTok{==}\StringTok{ "2"}\NormalTok{)] <-}\StringTok{ }\DecValTok{10}\OperatorTok{^}\DecValTok{2}
\NormalTok{storm.damage}\OperatorTok{$}\NormalTok{PropFactor[(storm.damage}\OperatorTok{$}\NormalTok{PROPDMGEXP }\OperatorTok{==}\StringTok{ "3"}\NormalTok{)] <-}\StringTok{ }\DecValTok{10}\OperatorTok{^}\DecValTok{3}
\NormalTok{storm.damage}\OperatorTok{$}\NormalTok{PropFactor[(storm.damage}\OperatorTok{$}\NormalTok{PROPDMGEXP }\OperatorTok{==}\StringTok{ "h"}\NormalTok{)] <-}\StringTok{ }\DecValTok{10}\OperatorTok{^}\DecValTok{2}
\NormalTok{storm.damage}\OperatorTok{$}\NormalTok{PropFactor[(storm.damage}\OperatorTok{$}\NormalTok{PROPDMGEXP }\OperatorTok{==}\StringTok{ "7"}\NormalTok{)] <-}\StringTok{ }\DecValTok{10}\OperatorTok{^}\DecValTok{7}
\NormalTok{storm.damage}\OperatorTok{$}\NormalTok{PropFactor[(storm.damage}\OperatorTok{$}\NormalTok{PROPDMGEXP }\OperatorTok{==}\StringTok{ "H"}\NormalTok{)] <-}\StringTok{ }\DecValTok{10}\OperatorTok{^}\DecValTok{2}
\NormalTok{storm.damage}\OperatorTok{$}\NormalTok{PropFactor[(storm.damage}\OperatorTok{$}\NormalTok{PROPDMGEXP }\OperatorTok{==}\StringTok{ "-"}\NormalTok{)] <-}\StringTok{ }\DecValTok{10}\OperatorTok{^}\DecValTok{0}
\NormalTok{storm.damage}\OperatorTok{$}\NormalTok{PropFactor[(storm.damage}\OperatorTok{$}\NormalTok{PROPDMGEXP }\OperatorTok{==}\StringTok{ "1"}\NormalTok{)] <-}\StringTok{ }\DecValTok{10}\OperatorTok{^}\DecValTok{1}
\NormalTok{storm.damage}\OperatorTok{$}\NormalTok{PropFactor[(storm.damage}\OperatorTok{$}\NormalTok{PROPDMGEXP }\OperatorTok{==}\StringTok{ "8"}\NormalTok{)] <-}\StringTok{ }\DecValTok{10}\OperatorTok{^}\DecValTok{8}
\end{Highlighting}
\end{Shaded}

So after replacing the factors, we can now calculate the economic damage
* we introduce a new variable called EconDam which is equal to total
economic loss due to crop and property damage by storms. For this, we
need to multiply the damage by the transformed factors thus * EconDam =
(CROPDMG x CropFactor) + (PROPDMG x PropFactor)

\begin{itemize}
\tightlist
\item
  then we have first filtered values which had some impact on economy
\item
  then grouped by Event type and finally
\item
  summarised by sum and arranged in decreasing order
\end{itemize}

\begin{Shaded}
\begin{Highlighting}[]
\NormalTok{Econ.Dam <-}\StringTok{ }\NormalTok{storm.damage }\OperatorTok\StringTok{ }\KeywordTok{mutate}\NormalTok{(}\DataTypeTok{EconDam =}\NormalTok{ (CROPDMG }\OperatorTok{*}\StringTok{ }\NormalTok{CropFactor) }\OperatorTok{+}\StringTok{ }\NormalTok{(PROPDMG }\OperatorTok{*}\StringTok{ }\NormalTok{PropFactor)) }\OperatorTok
\StringTok{      }\KeywordTok{filter}\NormalTok{(EconDam }\OperatorTok{>}\StringTok{ }\DecValTok{0}\NormalTok{) }\OperatorTok
\StringTok{      }\KeywordTok{group_by}\NormalTok{(EVTYPE) }\OperatorTok\StringTok{ }
\StringTok{      }\KeywordTok{summarise}\NormalTok{(}\DataTypeTok{EconLoss =} \KeywordTok{sum}\NormalTok{(EconDam)) }\OperatorTok
\StringTok{      }\KeywordTok{arrange}\NormalTok{(}\KeywordTok{desc}\NormalTok{(EconLoss))}
\end{Highlighting}
\end{Shaded}

\begin{verbatim}
## `summarise()` ungrouping output (override with `.groups` argument)
\end{verbatim}

So, now we may want to the economic value to be represented in million
USD, as we can expect that the Storm events may have caused huge
economic loss.

\begin{Shaded}
\begin{Highlighting}[]
\NormalTok{Econ.Dam}\OperatorTok{$}\NormalTok{EconLoss <-}\StringTok{ }\NormalTok{Econ.Dam}\OperatorTok{$}\NormalTok{EconLoss}\OperatorTok{/}\DecValTok{10}\OperatorTok{^}\DecValTok{9}
\end{Highlighting}
\end{Shaded}

Now, the economic values are conerted into billions USD, So, this data
is ready for our question.

\section{4. Results}\label{results}

So, now we can plot Health Damage against the storm event type, to see
the highest impact on Human health.

\begin{Shaded}
\begin{Highlighting}[]
\KeywordTok{ggplot}\NormalTok{(}\DataTypeTok{data =}\NormalTok{ health.Dam, }\KeywordTok{aes}\NormalTok{(}\DataTypeTok{x =} \KeywordTok{reorder}\NormalTok{(EVTYPE, }\OperatorTok{-}\NormalTok{values), }\DataTypeTok{y =}\NormalTok{ values, }\DataTypeTok{fill =}\NormalTok{ EVTYPE)) }\OperatorTok{+}\StringTok{ }
\StringTok{      }\KeywordTok{geom_bar}\NormalTok{(}\DataTypeTok{stat =} \StringTok{"identity"}\NormalTok{, }\DataTypeTok{width =} \FloatTok{0.7}\NormalTok{, }\DataTypeTok{color =} \StringTok{"black"}\NormalTok{, }\DataTypeTok{show.legend =} \OtherTok{FALSE}\NormalTok{) }\OperatorTok{+}\StringTok{ }
\StringTok{      }\KeywordTok{scale_fill_brewer}\NormalTok{(}\DataTypeTok{palette =} \StringTok{"Set3"}\NormalTok{)}\OperatorTok{+}
\StringTok{      }\KeywordTok{facet_wrap}\NormalTok{(}\OperatorTok{~}\NormalTok{variables, }\DataTypeTok{scales =} \StringTok{"free_x"}\NormalTok{) }\OperatorTok{+}
\StringTok{      }\KeywordTok{labs}\NormalTok{(}\DataTypeTok{x =} \StringTok{"Storm Event Types"}\NormalTok{, }\DataTypeTok{y =} \StringTok{"Total Number of affected Individuals"}\NormalTok{) }\OperatorTok{+}\StringTok{ }
\StringTok{      }\KeywordTok{ggtitle}\NormalTok{(}\StringTok{"Top Health Damage by Storm Events in US"}\NormalTok{) }\OperatorTok{+}\StringTok{ }
\StringTok{      }\KeywordTok{coord_flip}\NormalTok{() }\OperatorTok{+}
\StringTok{      }\KeywordTok{theme_bw}\NormalTok{()}
\end{Highlighting}
\end{Shaded}

\includegraphics{CourseProject2_files/figure-latex/unnamed-chunk-12-1.pdf}

This figures clearly shows that Tornados are most harmfuls storm events
in US. In both type of Human health impact, Tornados have the most
devastating effects on Human health

So, now we can plot Economic Damage against the storm event type, to see
the highest impact on Economy.

\begin{Shaded}
\begin{Highlighting}[]
\KeywordTok{ggplot}\NormalTok{(}\DataTypeTok{data =}\NormalTok{ Econ.Dam[}\DecValTok{1}\OperatorTok{:}\DecValTok{10}\NormalTok{,], }\KeywordTok{aes}\NormalTok{(}\DataTypeTok{x =} \KeywordTok{reorder}\NormalTok{(EVTYPE, }\OperatorTok{-}\NormalTok{EconLoss), }\DataTypeTok{y =}\NormalTok{ EconLoss, }\DataTypeTok{fill =}\NormalTok{ EVTYPE)) }\OperatorTok{+}\StringTok{ }
\StringTok{      }\KeywordTok{geom_bar}\NormalTok{(}\DataTypeTok{stat =} \StringTok{"identity"}\NormalTok{, }\DataTypeTok{width =} \FloatTok{0.7}\NormalTok{, }\DataTypeTok{color =} \StringTok{"black"}\NormalTok{, }\DataTypeTok{show.legend =} \OtherTok{FALSE}\NormalTok{) }\OperatorTok{+}\StringTok{ }
\StringTok{      }\KeywordTok{scale_fill_brewer}\NormalTok{(}\DataTypeTok{palette =} \StringTok{"Set3"}\NormalTok{)}\OperatorTok{+}
\StringTok{      }\KeywordTok{labs}\NormalTok{(}\DataTypeTok{x =} \StringTok{"Storm Event Types"}\NormalTok{, }\DataTypeTok{y =} \StringTok{"Total Economic Loss (Billion USD)"}\NormalTok{) }\OperatorTok{+}\StringTok{ }
\StringTok{      }\KeywordTok{ggtitle}\NormalTok{(}\StringTok{"Top Economic Damage by Storm Events in US"}\NormalTok{) }\OperatorTok{+}\StringTok{ }
\StringTok{      }\KeywordTok{coord_flip}\NormalTok{() }\OperatorTok{+}
\StringTok{      }\KeywordTok{theme_bw}\NormalTok{()}
\end{Highlighting}
\end{Shaded}

\includegraphics{CourseProject2_files/figure-latex/unnamed-chunk-13-1.pdf}

Thus, the above figure clearly show that Floods cause highest economic
loss in US, which correspons to more than 150 Billion USD. This amount
is almost double to the economic loss caused by Hurricanes or Typhoons.

\section{5. Conclusion}\label{conclusion}

In conclusion, Our data analysis clealry depicted that Tornaodos causes
highest damage to human health including fatalities and injuries whereas
Floods cause highest economic loss in US. Therefore, the government or
municipal should focus on preparing for severe weather events and will
need to prioritize resources for Tornados and Flood types of events.

\end{document}
